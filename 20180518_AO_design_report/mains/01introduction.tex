%---------------------Introduction----------------------------------
\chapter{Introduction} \label{chap:intro}
Les télescopes terrestres sont abondamment utilisés dans la découverte de notre Univers. Ces derniers bénéficies d'un filtre naturel grâce à l'atmosphère. Celle-ci permet d'atténuer les radiations lumineuses et donc de pouvoir étudier certaines zones de l'espace sans être saturées par la luminosité intense réfléchie par certaines étoiles. Cependant, cette même atmosphère engendre des inconvénients notables dans l'observation spatiale. L'instrumentation et le développement de techniques d'observation dans ce domaine doivent sans cesse améliorer la qualité de leurs constituants car les découvertes demandent d'aller de plus en plus loin avec de plus en plus de précision. Une avancée technologique a permise de considérablement perfectionner les télescopes du monde entier: l'optique adaptative. Cette technique permet alors de corriger les déformations de front d'onde en temps réel grâce à l'implémentation de miroirs déformables dans l'optique du télescope. Cependant, les aberrations statiques de l'optique installée après le système d'optique adaptative ne sont pas corrigées. Plusieurs méthodes ont alors été découvertes et mises en place afin de pouvoir améliorer la qualité des observations. L'une d'entre-elles est la méthode d'analyse de diversité de phase. Cette dernière permet de calculer les anomalies engendrées par le système optique en prenant des images à son plan foyer. Ces images sont alors traitées suivant un algorithme itératif qui permet de restituer la phase du front d'onde et d'en déduire les aberrations résiduelles. Néanmoins, les tests déjà réalisés sur plusieurs télescopes montrent que des erreurs contraignent la précision et donc la qualité des mesures astronomiques.\\

Ce document traite de l'application de la technique de mesure par diversité de phase dans le cadre d'une manipulation réalisée en laboratoire. Les effets dus aux aberrations atmosphériques qui s'additionnent à ceux de l'optique statique sont donc écartés. Ensuite, l'introduction d'une turbulence atmosphérique contrôlée permet de déterminer la précision et la qualité de mesure par cette méthode.\\

Le montage d'un banc optique recréant une source dont les caractéristiques sont connues et optimisées est tout d'abord réalisé. L'analyse des aberrations consiste en premier par la description de celles du système optique utilisé pour les mesures. Un faisceau lumineux issu d'une source ponctuelle est créé. Un système optique permet ensuite d'établir un front d'onde plat. Le faisceau traverse alors le système turbulent et est renvoyé sur une caméra CCD. Plusieurs tests sont réalisés afin de pouvoir déterminer avec quelle précision l'algorithme de diversité de phase retrouve les aberrations statiques.
Des mesures de bruit dus à la caméra permettent de caractériser la précision des futurs essais.
Puis des composants optiques dont les aberrations qu'ils engendrent peuvent être calculée sont insérés sur le banc. La détermination des aberrations par diversité de phase peut alors être comparée à celle calculée théoriquement et donc une estimation de la précision de la méthode peut en être déduite.
Pour finir, des mesures de diversité de phase en milieu turbulent sont réalisées afin de pouvoir évaluer la performance de la méthode dans des conditions plus proches de la réalité. 



